As defined by Bellazzini et al.\ (2013), the NGC~3109 filament is  distributed along a line extending from Leo P in the north ($\delta = +18^{\circ}$) to
Antlia in the south ($\delta = -27^{\circ}$). The galactic latitude ranges from $b = +54^{\circ}$ (Leo P) to $b = +22^{\circ}$ (Antlia).
We have mapped this filament onto sky positions and defined a series of 21 MegaCam
pointings which will cover the region from Leo P to Sextans A  ($\delta = -5^{\circ}$); 
the southern-most part of the filament is better-observed at the beginning of Semester A and we will re-propose next semester. % modify?

We propose to image in a single filter  and use spatial filtering algorithms and visual examination to detect overdensities along the filament direction,
following the method used by Chiboucas et al. (2009) for their successful detection of more than a dozen dwarf galaxies in the M81 group.
Candidate dwarf galaxies or tidal debris can then be followed-up with multi-band imaging over smaller areas, either with MegaCam or Gemini/GMOS.

Of the MegaCam filters, the $r$ band is best-matched to the SED peak of red giants in an old stellar population.
The depth is set by the requirement to detect enough stars for a significant measurement of overdensity. As a comparison,
with data reaching 3 magnitudes below the TRGB, McQuinn et al.\ (2013) found Leo P to have a {\em central} surface brightness of $\mu_V=24.5$~mag~arcsec$^2$ and only a few hundred RGB stars. We aim to reach 
a depth of two magnitudes below the TRGB ($M_r = -3.1$, or  $r=23$ at 1.7~Mpc), meaning $r=25$.  
To reach $S/N=5$ at this depth requires 0.5-hour exposures (including overheads for 4 dithered exposures) 
in relatively pessimistic conditions (grey time, 1 arcsec seeing, airmass 1.5; we note that Chiboucas et al.\ (2009) had somewhat
better conditions and reached $r=25$ in approximately 1200~s exposures).
We therefore request a total of 10.5 hours of exposure time.


The proposed observations  overlap in RA with Large Programmes MATLAS and LUAU. With this in mind, we have constrained the observing
conditions only loosely.

