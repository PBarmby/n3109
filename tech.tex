As defined by Bellazzini et al.\ (2013), the NGC~3109 filament is  distributed along a line extending from Leo P in the 
north ($\delta = +18^{\circ}$, $b = +54^{\circ}$) to
Antlia in the south ($\delta = -27^{\circ}$, $b = +22^{\circ}$); Sextans B is about 3$^{\circ}$ to the west of this line.
While galaxies belonging to the NGC~3109 association might not lie along the filament defined by
the 5 known members, the structure provides the most obvious `lamp post' under which to search for new
members. The southern part of the filament will be observed with LSST (coverage to $\delta < +10^{\circ}$ to the necessary depth),
but the northern portion is out of LSST's reach. The SDSS coverage of the northern portion of the filament
is of insufficient depth for our purposes.
We have defined a series of Megacam pointings covering a $2^{\circ}\times6^{\circ}$ region south
of Leo~P in the direction towards Sextans~A~and~B; see Figure 2. This area subtends $44\times136$~kpc
at the distance of the NGC~3109 association; for comparison, the tidal truncation radius of Sextans~A is
about 3~kpc (Bellazzini et al.\ 2014). As the Figure shows,  our defined fields are also
quite close to the more nearby (240~kpc) dwarf galaxy Leo~I and can be used to probe the outer
regions of this galaxy. The position of the field nearest Leo~I has been adjusted slightly to avoid the
bright star Regulus.


Megacam imaging in two filters is important for separating the giant branches of additional NGC~3109
members from both foreground Galactic stars and background unresolved galaxies.  Bellazzini et al.\ (2013, 2014)
successfully demonstrated the ability to effectively remove contamination using colour information
in an LBT study of Sextans A and B in $g$ and $r$. Of the MegaCam filters, the $r$ band is best-matched to the SED 
peak of red giants in an old stellar population.
The imaging depth is set by the requirement to detect enough stars for a significant measurement of overdensity. 
With data reaching 3 magnitudes below the TRGB, McQuinn et al.\ (2013) found Leo P to have a {\em central} surface brightness 
of $\mu_V=24.5$~mag~arcsec$^2$ and only a few hundred RGB stars. We aim to reach 
a depth of two magnitudes below the TRGB ($M_r = -3.1$, or  $r=23$ at 1.7~Mpc), meaning $r=25$;
this is shallower than the Bellazzini et al.\  data, but their fainter objects were dominated by background galaxies. 
The $g-r$ colour found by Bellazzini et al.\ at $r=25$ was $g-r\approx 1.0$ so the required $g$-band depth is $g=26$.

We will search for new, faint members of the NGC~3109 association using the standard methods in the field:
searching for spatial overdensities of resolved stars within magnitude and colour ranges expected for the stellar populations.
The resolved stellar population studies of known NGC~3109 group members provide excellent templates for
this work. Foreground and background (stellar and galaxy) populations are also well-characterized in these
studies and others, so filtering out only the populations of interest is feasible.

To estimate exposure times, we assume less-than-perfect conditions (dark time, 1 arcsec seeing, airmass 1.2, 90\% transparency)
since our proposed observations  overlap in RA with Large Programs MATLAS and LUAU and we have not
tightly constrained the observing
conditions. Assuming overheads for 4 dithered exposures per filter, reaching 
$r=25$ and $g=26$ at $S/N=5$ requires 1.5 hours per field (0.5 in $r$ and 1.0 in $g$.)
(We note that in their successful Megacam detection of more than a 
dozen dwarf galaxies in the M81 group, Chiboucas et al.\ (2009) had somewhat
better conditions than assumed here and reached $r=25$ in approximately 1200~s exposures.)
We therefore request a total of 18 hours of exposure time.


