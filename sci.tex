The large scale structure of our local universe is dominated by voids and filamentary structures of groups and clusters of galaxies 
(e.g., Courtois et al.\ 2013).  Half of the Local Volume galaxies reside in loose groups and associations, such as the Sculptor filament 
and the Canes Venatici Cloud (Karachenstev 2005). The nearest such structure is the NGC~3109 association (see Figure 1), an
 association of 5 dwarf galaxies in Leo, Sextans, and Antlia (van den Bergh 1999; Tully et al.\ 2006). This galaxy association, 
 with a linear extent of 1.2~Mpc across the sky (Bellazzini et al.\ 2013), lies just at the edge of the Local Group at a distance of 
 1.3--1.7 Mpc. The origin of the NGC~3109 association is discussed in detail by Pawlowski \& McGaugh (2014), who identify 
 several possibilities: an infalling dark matter filament, a  pre-existing group tidally stretched by an encounter with the MW, or galaxies
 formed as tidal dwarfs from a past Milky Way/M31 encounter. Constraining the number and properties  of additional group
 members can help to discriminate between these possibilities as well as confirm or refute the suggestion by Pawlowski \& McGaugh (2014)
 that {\em all} of the non-satellite galaxies to the north of the Milky Way are confined to a single plane.

The time is right for a search for additional members of the NGC~3109 association. While the region is covered by existing  and planned
survey data, none are deep enough to detect the faint galaxies and  structures at this distance. The most
well-known recent examples of nearby dwarf galaxy discovery are from the SDSS (e.g., Koposov et al.\ 2008).
However, despite lying within the SDSS footprint, the newest member of the NGC~3109 association, Leo P, 
was recently discovered as part of a blind H~I survey (Giovanelli et al.\ 2013): it is simply too faint to
be well-detected at the SDSS  depth ($g=22.2$, $i=21.3$)
%This galaxy has an H~I mass of $10^6$ solar masses and a stellar mass about $3\times10^5$~M$_{\odot}$, and is extremely metal-poor:
Using LBT imaging of Leo~P, McQuinn et al.\ (2013) measured the tip of its red giant branch to be at $i=22.1$.
Further south, tidal substructure near the Antlia dwarf suggestive of an interaction with NGC~3109 was recently discovered by Penny et al.\ (2012),
at very low surface brightness levels.
This suggests to us that additional  faint members of the group as well as tidal debris from interactions between group members 
may await discovery.

The outcome of a search for additional members of the NGC~3109 association has important implications for
understanding galaxy evolution. A lack of additional tidal debris could cast doubt on the `tidal dwarf' 
explanation for the group's existence, while a lack of additional bound-galaxy members of the group 
would strongly constrain the mass of a possible dark-matter filament or progenitor halo
On the other hand, finding additional members of the group would allow a better characterization of its
spatial and dynamical extent; with only five currently-identified members, such studies 
suffer from small-number statistics. If additional group members lie close to the same plane
as the existing members, this would tend to confirm the result of Pawlowski \& McGaugh (2014). 
Identification of new members as potential `backsplash galaxies' (Teyssier et al.\ 2012) might lend credence to 
 Pawlowski \& McGaugh's
claim that there is an `overabundant backsplash galaxy' problem with $\Lambda$CDM.
Additional group members 
would also allow better constraints on the group's  luminosity function and subsequent comparison to CDM 
simulations of galaxy formation in  low-density environments.

%A discussion of how the observations will constrain the underlying theory is also missing: are additional (presumably HI-deficient) satellites or stellar streams expected theoretically, and what are the implications if they are or are not found? 


% Filamentary structures are considered to be in an earlier stage of their evolution, thus a complete characterization of the interacting environment of the NGC~3109 structure will provide us with insights to lend interpretation to more distant structures, such as the Sculptor filament (e.g. Jerjen et al.\ 1998).

The southern part of the NGC~3109 filament, from NGC~3109 to Sextans B, is expected to be observed with LSST.
The northern portion of  the NGC~3109 filament, from Sextans B to Leo P, requires deeper imaging than
the available  SDSS  depth ($g=22.2$, $i=21.3$) in order to detect dwarf galaxies at distances beyond 1 Mpc.
CFHT/MegaCam's ability to go both wide and deep is essential for such observations: the nearby nature of the 
filament means that it covers a large area of sky, yet the expected faint dwarf galaxies are resolved into individual stars.
Completing the substructure census and probing the faintest end of the luminosity function in the nearest filamentary 
structure are the goals of this proposal. Detected low-surface brightness structures will be characterized in terms of 
structural properties and their association to the sub-group will be pursued via tip-of-the-red-giant-branch 
(TRGB; Lee et al.\ 1993) measurements and, if possible, kinematical measurement.


% Begyu et al.\ 2013, AJ 145, 120\\ % "AN INTERACTING GALAXY SYSTEM ALONG A FILAMENT IN A VOID "
%Bellazzini, M. et al.  2013, A\&A 559, L11\\
%Chiboucas, K. et al.\  2009, AJ, 137, 3009\\
%Courtois, H.R. et al.\ 2013, AJ, 146, 69\\
%Giovanelli, R. et al.\ 2013, AJ, 146, 15\\
%Jerjen, H. et al.\ 1998, AJ, 116, 2873\\
%Karachentsev, I. 2005, AJ, 129, 178 \\
%Keller, S.C. et al.\ 2007, PASA, 24, 1\\
%Koposov, S. et al.\ 2008, ApJ, 686, 279\\
%Lee, M.G et al.\ 1993, ApJ, 417, 993\\ 
%McQuinn, K.B.W. et al. 2013, AJ, 146, 145\\
%Penny, S.J. et al. 2012, ApJ, 758, L32\\
%Tully, R.B. et al.\ 2006, AJ, 132, 729\\
%van den Bergh, S. 1999, ApJ, 517, L97\\

%\clearpage


% get these ordered correctly at end
\section*{References}

\begin{table}[h]
\begin{tabular}{ll}
Bellazzini, M. et al.\  2013, A\&A 559, L11 & Pawlowski, M.S. \&  McGaugh, S.S. 2014, MNRAS, 440, 908\\
Bellazzini, M. et al.\  2014, A\&A 566, 44 & Koposov, S. et al.\ 2008, ApJ, 686, 279\\
Chiboucas, K. et al.\  2009, AJ, 137, 3009 & Lee, M.G. et al.\ 1993, ApJ, 417, 993\\
Courtois, H.R. et al.\ 2013, AJ, 146, 69 &  McQuinn, K.B.W. et al. 2013, AJ, 146, 145\\
Giovanelli, R. et al.\ 2013, AJ, 146, 15 & Penny, S.J. et al. 2012, ApJ, 758, L32\\
Jerjen, H. et al.\ 1998, AJ, 116, 2873  &  Tully, R.B. et al.\ 2006, AJ, 132, 729\\
Karachentsev, I. 2005, AJ, 129, 178 &van den Bergh, S. 1999, ApJ, 517, L97\\
Keller, S.C. et al.\ 2007, PASA, 24, 1 & Sohn, S.T. et al.\ 2007, ApJ, 663, 960\\
{Pe{\~n}arrubia}, J. et al.\ 2009, ApJ, 298, 222\\
\end{tabular}
\end{table}



\begin{figure}
\caption{The NGC 3109 association shown in Cartesian coordinates centred on NGC 3109. From Bellazzini et al.\ (2013).}
\includegraphics[scale=0.4]{bellaz_fig1}
\includegraphics[scale=0.4]{fields_view}

%\caption{Equatorial coordinates of the NGC 3109 association galaxies. The present proposal is to image a 2-degree-wide strip from Sextans B to Leo P using MegaCam.}
%\includegraphics[scale=0.7]{n3109_group_sky}

\end{figure}

%\begin{figure}
%\caption{Close-up of the northern part of the NGC~3109 association, showing the proposed MegaCam fields in a  2-degree-wide strip from Sextans B to Leo P.}
%\includegraphics[scale=0.8]{fields}
%\end{figure}


