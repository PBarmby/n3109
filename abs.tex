The NGC 3109 association is a nearly-linear group of dwarf galaxies located at the edge of the Local Group (d$\sim$1.4 Mpc). The origin of this group is unclear: is it an infalling dark matter filament, a set of 'backsplash' galaxies which have passed through the Milky Way, or a group of tidal dwarfs formed in a past Milky Way/M31 encounter? Distinguishing among these possibilities requires a complete accounting of the  galaxy members of the association followed by characterization of their star formation histories and kinematics. CFHT/MegaCam provides the ideal capabilities to begin a detailed search for additional galaxies.
The results of such a search will test a recent prediction from simulations that the Local Volume should have at least a dozen more dwarf galaxies than
are currently observed. The proposed MegaCam imaging will also serve an additional purpose: it can also be used to search for the tidal break and other low-surface-brightness features around the  foreground (240 kpc) dwarf galaxy Leo I. 
